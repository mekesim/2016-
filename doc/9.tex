\chapter{第九期}

\section{2021年9月1日 周三 晴}

陈思涵

20世纪30年代,在英国一个名不见经传的小镇里有一个叫玛格丽特的小姑娘。她的父亲经常向她灌输这样的观点:无论做什么事情都要力争一流,永远走在别人前头,而不能落后于人。“即使坐公共汽车,你也要尽量坐在前排。”父亲也不允许她说“我不能”或者“太难了”之类的话。这个女孩,就是英国的第一位女首相---玛格丽特·撒尔切夫人。

在我们的日常生活、学习中,也应该像她一样抱着一往无前的精神和必胜的信念,尽自己最大努力克服一切困难,做好每一件事,事事必争一流,以自己的实际行动践行“永远坐在前排”。

\section{2021年9月6日 星期一 晴}

戴瑞彤

今天我做好黑板报,已是傍晚时分。天渐渐暗了下来,太阳就像一个光芒四射的大火球,照得我都睁不开眼睛,把世间万物都染成了金黄色。过了一会儿,太阳变得柔和了,变成了温柔的橙色,就像一个巨大的橙子一样,正慢慢地往下掉。太阳又成红色的了,犹如一盏圆圆的橙红色灯笼,也如同一颗红色的大宝石,浑身散发着红色的美丽光芒,可爱极了。

很快,我就回到家了,正想做作业呢,结果发现笔袋不见了,兴许是落在教室了吧。

\section{2021年9月7日 周二 阴}

孙轩睿

今天爸爸回来得比较早,于是他便和妈妈杀了一盘军棋。他们玩的是军棋中暗棋这种玩法,这种玩法就是不能看对方的棋,吃子时也不能看,所以需要一个裁判,当然这个裁判就是我。
刚开始,爸爸走错了好几步棋,被妈妈连吃数子。过了一会儿,爸爸的司令很厉害,连斩数将,妈妈可能以为那是中等大的棋,一直用小一点的子去试,当时,我真想告诉妈妈那是一个司令,用炸弹或司令对掉。可是我现在是一名裁判,并不是一名观众,而且就算是观众也不能说,一定要让他们公平地对决。思想斗争了一番,我还是决定不告诉妈妈,让他们好好杀一盘。过了一会儿,妈妈终于用司令和爸爸的司令同归于尽。
作为一名裁判,一定要做到公平、公正、公开(虽然暗棋不能告诉对方),不能作弊,如果你作弊了,你也就失去了一样重要的东西——诚信。

\section{2021年9月8日 周三 小雨}

陈思涵

这个暑假也许是我最难忘的一个暑假了。我参加了钢琴十级的考试。而且因为上个学期没怎么练,压力倍增。我也曾想过放弃,可是每当我想到放弃时,仿佛总有一个声音在我耳边回荡:“你既然选择了考级,那么无论多困难,你也得继续坚持!“我若有所思地点了点头,心想:”做事就要做成,不能半途而废。如果我现在选择了放弃,那么前面所付出的一切都付诸东流了。“我咬了咬牙关,望着乌黑的钢琴,坚定地点了点头,继续练了下去。

虽然不知道结果如何,但我相信,我的努力总会发光!

\section{2021年9月13日 星期一 雨}

戴瑞彤

终于等到了饭点,时钟上的指针跑到了12点的位置,我迈着轻松而又愉快的步伐向食堂走去。好不容易可以吃饭了,哇,今天中午有我喜欢吃的牛排,我赶快拿起勺子,向牛排切了下去,顿时,肉汁奔出,外焦里嫩,尝一口下去,回味无穷。当我吃得正欢之时,一只苍蝇竟飞到了我的牛排上,我用力往牛排上吹气,可那苍蝇却纹丝不动,我拿起饭盒,左右摇了摇,没用,用手拍拍放牛排的那块格子,还是没用,看来今天吃牛排的计划泡汤啦!

\section{2021年9月13日 周一 小雨}

宋旻峻

今天阳光体育果结束之后,我们就放学了。我拿着硬币,在车站等公交车。正等着车呢,我的肚子就开始叫了。看着其他同学都吃着零食,我和一旁的李朗都饿了。“要不我们去买包子?”我提议。“好耶!干饭去喽!”我们立刻向包子店冲去。这时,电话响了。“喂?老妈,什么事?”“你在哪儿啊?我来接你了!”哎,没办法,只能和包子说“拜拜”了。老妈,你的这电话响的可真不是时候啊!

\section{2021年9月14日 周二 阴}

孙轩睿

电脑课下课了,我趴在桌子上发呆。我看到宋旻峻轻轻地打了一下范思雯,她说:“嗯?什么东西?”我还以为她在说宋旻峻呢!结果扭头一看,哦买嘎!一只大大的蜂(应该是虎头蜂)在窗户顶啊顶,有一些同学拿着书就冲上去打它,在这令人兴奋而又紧张的时刻,有人说:“老师来了。”马上又有人说:“不不不,老师没来。”一时间,大家都不知道要不要回座位,我出去看了一眼,老师没有来。从窗外看,教室内的战况更激烈了。一帮人围在叶雨飞座位边,在打那只虎头蜂,蒋鲁弋在最前面大喊:“盖亚!”使劲挥动书本。
后来虎头蜂不知被谁打死了,这场战斗才停止。

\section{2021年9月16日 周四 晴}

陈思涵

我给自己安排的作息表是每天早上5:30就要起床耗耗腿、开开功。可是每当我听到那串清脆的铃声,总会用手揉揉惺忪的睡眼,然后满怀怨恨地用力地拍停了闹钟,然后又一睡睡到6:30.可在昨天,闹钟一响,我就立马直起了身体,迅速下了床,点亮了台灯,赶紧开始“早自习“。不过话说回来,一个人在晨光中独自学习的感受还是不错的,希望下次可以依旧保持自觉。

\section{2021年9月17日周五 晴}

宋旻峻

今天老师在布置语文作业时,发生了一件很有趣的事,老师叫我们翻开家校联系本,写上语文作业:点面。点面?唉?这是什么奇怪的东西?难不成,还是让我们去餐馆点一碗面吃?啊啊,那可太好了,我的脑子里立刻想象出一碗香喷喷、热呼的片儿川。可是此时老师解释是用点面结合的方法写小练笔。唉,美梦破灭了!

\section{2021年9月22日 周三 晴}

陈思涵

前几天在上舞蹈课的时候老师给我们安排了新的技巧——吸撩推大跳。一开始,大家都跌跌撞撞的,跳得乱七八糟的。老师在一旁,时不时为我们纠出错误,并告诉我们正确的做法。我也曾想过放弃,但老师一直在鼓励我们认真练习就一定会有结果。在经过了高强度的训练之后,我们都成功了!是啊,就像董卿老师说过的一句话:“人在这个世界上,无论选择哪一条道路,它都是荆棘与鲜花同在,有晴空也有冷雨。不过,就像鲁迅先生说的,前途很远、很暗,然而不要怕,不怕的人前面才有路。”

\section{2021年9月27日 周一 晴}

陈思涵

教室里,同学们都在互相拉伸。紧张的氛中又稍带了些许的轻松,时不时还传出几声嚎叫:“你轻点儿啊!帮我压一下好不好啊……”椅子被统一推进桌底下,过道上挤满了人,同学们三个一伙,五一个群不留余地地在过道上拉伸,有的生站在讲台上,弯下身子,不停地手去勾地面;有的女生坐在别人背上,使出全身力气往下压;而我,直接干脆了当地搬来一张凳子把一条腿架上去,开始压压筋。眼看着就要轮到我测试了,虽然我在班理里算软的,但我也不能掉人轻心觉得自己一定满分,毕竟在家练的成绩也不好。

排在垫子后的我很快就轮到了。我的心里其实也很紧张,心扑通扑通上下直跳。脱鞋子的时候,我的手都是颤抖的。不知为何,我刚坐垫子,旁边就拥上来了一群同学,她们一个个都挤来挤去,好多脑袋从人群中探出来。有个生对我说:“你让让我吧!”还有的女生在一旁给我鼓劲,还有的女生似乎在等待着什么发生,满脸期待地望着我。我深吸了一口气,把腰挺得笔直笔直的。猛的一下用力推动游标,直到我无法再往前推一点点。我抬起头,等待着孙老师报出我的成绩。明明只有短短的一秒钟,可我却觉得十分难熬,仿佛度过了一个世纪。孙老师低头看了看板面,说:“28!”顿时,我心里吊着的一块大石头终于落了地。

\section{2021年9月27日 周一 晴}

许智涵

无人机第一轮比赛开始了,各位选手都拿出了最高的水平。丹枫小学的一位选手起飞后快速空翻、钻拱门、过遂道,却在最后一圈撞上隧道壁,而导致超时。另一位选手刚起飞就飞掉了螺旋桨;另一学校的选手从起飞到降落全程稳稳地飞,飞出了满分的好成绩。当比赛来第二轮,比赛场地稍微宽了一点,选手们好像也更兴奋了。一位选手起飞后直奔拱门,飞完两圈只用了大为50秒,飞机最后下降时“跳”了一下,飞出了得分区,没获得满分;另一位选手全程加速,却跌跌撞撞,翻倒了好几次,第一圈刚飞完就超时了。到我们学校了,李炫增增第一个,飞得比上一轮好,我第二个,却因为电池没电了而浪费了许多时间。其他选手都发挥地不错,共五个人获奖。

\section{2021年9月27日 周一 晴}

戴瑞彤

一开始,双方的“战争”还算和谐,绳子中间的红带子几乎没有移动,但渐渐的,那些男生要动真格了,他们露出真面目。男生们用力把绳子佳后拽去,不听话的绳子和红带带子正以肉眼可见的速度向男生那儿溜去,天平向男生那儿倾斜。

我们女生也赶忙稳住下盘,弯成一把弓的样子,努力把绳子往回拉。我的险热手乎的,看向我的队友们,她们虽然劳累,但还在咬牙坚持着。叶雨飞的脸红的如同苹果一般,鼻子里一直冒着粗气。来欣年的额头上已经出现了很多大颗的汗珠,那些汗珠顺着她的脸颊滑落到地上。姚琪的手上青筋暴起,脚紧钩着地面,用尽力气把绳子拉过来。围观的同学们也为我们加油呐喊,这鼓有力量的声音传遍了整个操场。

\section{2021年9月27日 周一 晴}

李卓涵

老师宣布了规则:“男女生内部分成两组进行比赛,三局两胜,基本规则我就不细讲了。”同学们都跃跃欲试,气氛一下活跃了许多。

先是女生比赛,她们势均力故,比赛变成了拉锯战如火如荼地进行着。看得我们像打了兴奋剂似的,个个都想上去大展身手,同学们都不坐着了,全站了起来。

男生拔河比赛开始了!老师吹响哨子,我们拼命往后拉赢了第一局。然后第二局我们大意失荆州,输掉了比赛。我们聚在一起讨论对策,可热闹了。有人说:“对面体力好,我们力量大,我们体力剩得都比较少了,第三局只能智取。”大家都同意他的说法。我想了想说:“我们可以先跟他们对抗,然后我喊三二一大家一起松手,他们就很有可能倒地,我们再拉,就有可能赢。”大家讨论了一会儿,同意了。第三局一开始我们拼命拉,建立了优势,确保不会一下被他们拉过去,我大声喊:“三!二!一!放!”我们一起松开了手,对方顿时倒地,我们像是几股绳拧在了一起变得很有劲,一下子就赢了。

\section{2021年9月27日 周一 晴}

姚琪

今天的校园格外热闹,大家都在出为体测作准备。每个地方的队伍都十分整齐,放眼望仿佛是一条条长龙在缓慢地爬行。“嘟——”随着一声哨响,体育馆一个个高低不同的人一齐开始跳绳,绳子在空中划出了一道直优美的曲线,显得朴实而又华丽。仰卧起坐只见人人都卖力地做着,却看不清是谁,也不知做了几个。50米的跑道上,一道道“残影”飞驰过去,不一会儿就都到了终点,在红色的跑道上留下了一道道蓝光;而在绿茵茵的草地上,白黑相间的足球在少年们的腿下行动自如,似乎有了生命。咦?是什么让一群同学围在教室门口?哦,原来是体前屈呀,大家把仪器推得“咻咻”直响,惊叹声时不时传入耳。

\section{2021年9月29日 周三 晴}

陈思涵

前几天阳光体育的时候,我在一旁休息。董迈被老师叫去做仰卧起坐。我就看着她奋力地做着。不一会儿,伴着一声清脆的声音,一枚硬币从她的口袋里“蹦”了出来。她也没管,继续做着。突然,有一个红红的东西从口袋里探出了头,落在了垫子上。我凑近一看,原来这团鼓鼓的东西是双袜子啊!不经意间,又有一张纸从口袋里钻了出来。我想,董迈的口袋怕不是哆啦a梦的口袋吧?什么都啊,真的太神奇了!我不禁佩服自己的想象力,在旁边忍不住笑出了声。

\section{2021年10月11日 星期一 阵雨}

戴瑞彤

俗话说“美好的时光总是短暂的”,我有亿点点赞成这句话,这不,这十天的国庆假期仿佛一眨眼就从我们的眼皮底下溜走了。

开学第一天,最重要的是什么?当然是收作业啦。一个同学接着一个同学,一连四五个同学都排队走向了讲台。“詹老师,我的好书推荐卡没拿来。”“詹老师,我的故宫一日游讲解稿不见了,我记得我放进书包了的。”等会儿,讲解稿?今天下午就发生了一件有关演讲稿的趣事,下篇闲言碎语带你们揭晓哦!

\section{2021年10月1日 周一 雨}

黄赫茗

前几天去看了电影《长津湖》,还挺好看的。但最后影片结束后,我有点失望,因为我没看见“彩蛋”。我觉得总得有个“尸骨回归”什么的呀。

当我看完电影起身时,听到有人说“走出影院后,你就会知道“彩蛋”是什么了。”可我走出影院却什么也没有发生……

后来我才领悟到:当你走出影院后,看到现在的万家灯火、繁华街道,都是那些前辈们为我们争取来的呀!

我望着这盛世,默念了一声:“谢谢你们。”

\section{2021年10月12日 周二 阴有时有雨}

孙轩睿

最近,我发现一件事,我们身边的某些文具可以完美地诠释了一些科学道理。、
就比如折叠尺,你只要一手拿住它展开一端,一手拿住另一端,将这把尺子快速地关上拉开、关上、拉开,注意这个动作不能停,要连续不断一直做。做了一小会后至少半分钟,你可以停一下,用手摸一下,这把尺子的两段一直连在一起的那个可以动的交界处,如果没有发热,请再重复以上动作,如果发热了,恭喜你,成功了。对比一下之前,你会发现它变热了。这是什么?摩擦生热!

生活中处处有科学,期待你的发现。

\section{2021年10月13日 周三 雨}

陈思涵

今天英语课上,蔡老师出了一个问题,问我们一个学中文的外国人需要买一本怎样的字典。下课后,孙轩睿拍拍我的肩膀:“外国人学语文为什么要字典呢?直接先送他一本五年高考,三年模拟得了!”我笑得差点岔气:“那还得先送一本五年中考,三年模拟呢!”他说:“时代在进步,高考在变难。说不定马上就要变成五年高考,百年模拟了!”我说:“哪有那么夸张啊,百年模拟怕不是在比寿命而不是比分数了!”

\section{2021年10月18日 周一 晴}

宋旻峻

今天的最后一节“阳光体育”课,老师让我们跳长绳。但是距离上一次跳仿佛恍如隔世,早就忘记怎么跳了。“什么是长绳?我不知道啊。”我装作一副傻傻的样子,逗得大家哈哈大笑。不过还好,刚跳几下就想起来了。不过在我前面的同学跳的时候总要伸开双手,经常绊到。“啊?难道这就是传说中的大鹏展翅吗?”在大家都绷着脸认真跳绳的时候,我冷不丁插上这么一句,紧张的气氛瞬间就被打破了。

\section{2021年10月20日 周三 雨}

陈思涵

人们常说:“活在当下。”确实,当下即过程。我们已无法改变定格的过去,但可以撰写以后的人生,难道不是吗?享受当下的同时,其实你已经享受了过程,过程就是结果。我们都想变的更好,那就应该为此努力啊,向前跑起来啊!所以,每个人都该为当下而拼搏!

\section{2021年10月20日 周三 中雨}

黄赫茗

在一片黑漆漆的废墟中,一个小男孩在地上寻找着什么。突然,一颗红色的曙光从天空中坠下。小男孩马上抄起地上的书包,朝着那道光坠落的方向跑去。不一会儿,小男孩背着一大包零件走了回来。当他走到一座火箭旁边时,他停了下来,火箭的旁边有一位老者,好像在研究着什么,那位老者见小男孩回来了,高兴的不得了。连忙拿过小男孩的包开始翻找,可什么需要的零件都没有发现,因此老者很失望。

老者给了小男孩一张图纸,可小男孩并不知道这是什么。正当小男孩思索时,突然又一道红色的曙光闪过…… 欲知后事如何,且听下回分解。

\section{2021年10月25日 周一 晴}

宋旻峻

今天放学后,我依照惯例先坐车到水印城,再转车回家。当我从水印城下车后,发现公交车已经在红绿灯路口了,而且现在是绿灯!“NO,NO,NO!”我心里大叫着,朝公交车站飞奔而去。我使出了九牛二虎之力,但却因为背着书包,始终跑不快。我的呼吸越来越急促,四肢也开始酸痛…… 终于,我看到了“距离车站20米”的标牌,我仿佛被加了一把火,速度也变快了。“哔,哔,哔……”在远处的车站传来关门的声音。“不要啊!”我的心态彻底崩了,喘着粗气蹲了下去。唉?等等,这不是我要坐的车呀!太好了!此时,我要坐的那一路车飞驰而来,我便欢天喜地地上车了。

\section{2021年10月25日 周一 晴}

黄赫茗

昨天10月24日,是程序员日。为什么要把程序员日定在10月24日呢?因为1024对于程序员来说是个很敏感、具有特殊意义的数字。1024B=1KB,1024KB=1MB,1024MB=1GB…… 还有一个原因是因为程序员的眼中只有二进制,也就是10110011010……

对了,昨天还是毛岸英和彭德怀的生日。

\section{2021年10月26日 周三 晴}

陈思涵

昨天上午早自习的时间,我不小心把宋旻峻弄伤了。当时我的心里翻江倒海,害怕极了:“我居然把同学弄伤了,老师家长会不会因此而讨厌我啊?我为什么那么不小心呢,要是刚才稍微关慢一点,就不会这样了……”我的眼泪差点就要流下来了,正当这时,老师走进教室了,我的内心都要打起架来了,是跟老师说还是隐瞒呢?如果说了,老师会不会批评啊,但如果不说,瞒得过初一瞒不过十五,老师总会知道的…… 最后我还是和老师说了,毕竟是我做错了就得承担后果。

\section{2021年10月26日 周二 晴}

杨若君

这天,我像平常那样,回到家里开始写作业。一开始,一切都进行的十分顺利,直到我开始做数学作业时出现了问题,因为有一道题超极难,爸爸给我出的这道题完全超出了我们所学的内容,我开始犯愁了,左手撑在桌子上,挠着脑做袋,右手拿着笔,皱着眉头,还咬着笔头。过了一会儿,笔里面的墨水。一不小心被我给吸了上来了,我张着嘴,生怕不小心把墨吞下去。我到厨房门口看了一下,确认爸爸有没有看见我,再三确认后,我大步流星地跑进侧所,使用飞一般的速度清理掉每一滴墨,然后又假装若无其事地写起了作业,可心里却还是翻江倒海的。看来以后写作业还要认真点儿,不能再咬笔了。

\section{2021年10月26日 周二 晴}

孙轩睿

体育课来了,我们总是在下楼前问华老师,今天体育课干什么,今天也不例外,没想到,华老师居然回答:“打篮球。”我们一阵欢呼,没想到华老师又邪魅一笑,补上一句:“测完跑步可以打。”我们心生疑惑:“是50米吗?”“不,太少了。”华老师说。有人问:“100米?”华老师笑着说:“加个0。”“什么1000米!”要累死人啊。人群中传来一声惊呼。华老师这才道出正确答案:“500米,500米。”“啊,那么多。”大家都像泄了气的皮球,垂头丧气的。

500米太累了,跑完都没力气打篮球了,华老师就是用篮球引我们好好跑500米吧。

\section{2021年11月1日 周一 晴}

宋旻峻

今天最后一节课是光体音,老师叫我们赶紧收拾,出去排队。我手忙脚乱地收拾打包,眼睛就要收拾好了,可偏偏阅读纸不知趣,跟我玩起了“捉迷藏”我急得满头大汗,只好本一本一本搜索。“要是让我找到,一定没你好看!”我一边喃喃着,一边一本一本地找:语文书、数学书、阅读纸…… 唉?等等,阅读纸?我连忙把它塞进包里,去排队去了。

\section{2021年11月1日 周一 晴}

黄赫茗

“人民”一词在很多政府建设的地方出现过,特别是在《宪法》、《民法典》上。有很多人以为“人民”是指除政府要员以外的所有人。其实“人民”是指除主席以外的所有人。你想想,习主席为了这中华大地的14亿人口操了多少心。毛泽东曾经也说过“为人民服务”,可以看出毛泽东笔下的“人民”二字,就是指除自己以外的所有中华儿女。

\section{2021年11月4日 周四 阴}

陈思涵

马上就要双十一了,我往购物车里加了好多东西。长长的列表,却引发了我深深的思考:“这么多快递,又要忙坏一群快递员了。哎…… 他们每天四处奔波,好不容易送一单却又很可能被打差评……”都说现在人人平等,不分高低贵贱,可很多人却因为种种原因,被他人“踩踏”与脚下,这分明是对人格的侮辱!我们应该尊重身边每一个人。“没有经历过,就没有发言权。”

\section{2021年11月8日 周一 晴}

宋旻峻

前几天是周末,我都嗨到很晚,几天都没睡好。所以今天早上我睡醒的时候,我还以为是周日,赖着死活不起来。“嗨!起床了!不要睡了!”妈妈大声地催我起床。唉呀,难得周末。”我迷迷糊糊地说,“就多睡会儿吧。”“什?周末”妈妈生气地叫着,“你自己看看!”说着,把手机举到我面前。我勉强睁开糊住的双眼,看了一眼:可不是嘛,11月8日,周一。我连忙陪着笑脸,一个咸鱼羽身飞一般起床,再飞一般地去洗漱。

\section{2021年11月8日 周一 晴}

黄赫茗

最近有很多武装汽车,经过钱江三桥下,这是为什么呢?就是因为台湾省的某些人搞了一些事情,都想要打仗了。

我有点忐忑不安,因为我知道,团结能使世界变得更美好,但战争可以导致整个国家甚至整个世界都灭亡。

我默默的许愿,愿战争不要发生……

\section{2021年11月9日 周二 晴}

杨若君

今天我放学回家时,抬头望着天空,心中不惊赞叹:啊!今天虽然很冷,可天空依然是那么美丽动人啊!看着这淡蓝色的天空,感觉瞬间让人的心情变得份偷快。你瞧,好像有一只小狗和一只小猫正嬉戏打闹着,抱着一个小球;又好像远处有一条小鱼,在河水里快乐地玩着,还可溅起阵阵水花…… 忽然,一辆飞机快速驶过,天上立刻出现了一条白线,仿佛是老奶奶织毛衣的毛线滚走后,留下的毛线。

今天的天空可真美可!

\section{2021年11月9日 周二 晴}

孙轩睿

学校像是“改革”了似的。

最近几周,学校里水果多了几种以前没在学校里吃过的,如:姑娘果,红心火龙果,猕猴桃等。

今天的水果是红心火龙果,吃起来比其他水果麻烦,要剥开来,用勺子挖着吃,不过好在味道很好,挖一下可以挖出很多,可以直接来一大口,很爽,而且那个果肉很甜,很好吃。由于是红心的,吃完以后满嘴红色,原来白白的牙变成了红色,让人联想到电影里的吸血鬼,就像喝了血一样。但在我们眼里就感觉好笑。

我对学校饭菜水果还是比较满意的。

\section{2021年11月10日0周三 晴}

陈姿伊

今天我妈妈下班回家后,狗子一看到她就蹭了上去,还露出了大肚皮。妈妈看着狗狗主动让她摸,非常高兴,于是立刻蹲了下去,扶摸着的它的脸蛋,另一只手给它抓痒,一边还说看:“唉呀,怎么会有这么可爱的狗子啊!”可谁知,没过一会儿,它突然一个鲤鱼打挺,“嗷呜”一声蹦了起来,还张着大嘴要咬人!吓得妈妈赶紧跑开了,还一边吐嘈,狗子脾气大,而我和外婆之前一直在“看戏”,所以顿时哈哈大笑起来。

\section{2021年11月22日 周一 晴}

黄赫茗

秋日已过,冬日已到,可树叶还是会有绿色的。比如校门口的柚子树,才黄了几片叶子,似乎寒风根本不会对它有影响。可旁边的树呢,都没有了绿色,也没有了黄色,红的地方却不少……

\section{2021年11月25日 星期四 晴}

杨若君

这天,我正在放学回家的路上。我走进了一条从未走过的小路。这条小路通向一个开满鲜花的公园,四周全是小花小草,树木也是长得高大粗壮,环境十分好,我一边走边哼着小曲儿。突然,一只大老虎出现在我的面前,我吓了一跳,愣了两秒后退了几步,虎老虎一点一点地逼近我,我害怕极了,吓得瑟瑟发抖。我灵机一动,心生一计:只要我躺在地上假装已经死了;可是要是它以为我真的死了,直接把我吃子可怎么办?算了,现在也只有这个办法了。于是我躺在地上,老虎不敢放松警惕,慢慢地靠近我,用爪子拨弄着我,我不敢睁开眼睛。突然,老虎大吼一声,我猛地睁开眼睛,只见妈妈正说着:“快起床了,都迟到了!”原来这只是一场梦啊,真是虚惊一场。

\section{2021年11月25日 周四 晴}

陈思涵

这次“闻涛杯”竞赛,给了我一个很大的巴掌。这次的语文总体还算简单,可是还是没有拿到心仪的分数。有道单选题,做过一遍还错,真是太不应该了,还有两道阅读选择题,文中有答案都没有选对!这只能说明在做的时候估计走神了,这么简单三道题还错。
唉,抱怨、后悔、羡慕都是没有用的,我应该调整好心态,认真面对下一次。

\section{2021年11月29日 周一 雨}

宋旻峻

这天放学时,我坐公交车回家。在快到家的前几站时,上来了一个小伙子坐在我旁边。他穿着脏梦兮兮的衣服,驼着背,跷着大二郎腿,还着着手机。时不时还用无神的眼睛神经兮兮地扫视四周,真叫人讨厌。我要下车了,他的二郎腿还如同拦路虎般翘在那里。“小朋友,你是要下车吗?”这时,一个温和的身声音传了过来,居然是那个小伙子!说完,他连忙撤出座位,还微笑着做出一个“请”的动作。哦,没想到他这么温和,人不可貌相呀。

\section{2021年11月29日 周一 雨}

盛夏颖

我可以说,体育课跑步简直就是要了马涵悦和丁凝的命。至于为什么,嗯…… 继续看下去吧!

每次华老师一说要和XXX班比赛时,她们俩就会……

马涵悦同学:飞速逃跑,趁大伙儿一个不留神儿,哎嘿,马上逃到二楼女厕所。让大家放弃找她,就此逃过接力。(但每次都被叶班等人抓到)

丁凝同学:嗯…… 虽和马涵悦同学逃跑路线一样,但灵活!也很勇敢!四处都敢去!(叶班有些时候也抓不到她吼!!!)

女生组结论:其实也没有啥子东西,只是…… 逃跑速度比跑步速度还要快罢了(但是还是跑不过叶班哦)感谢各位友情出演本闲言碎语。

\section{2021年11月30日 周二 晴}

杨若君

今天早上,我一睁开眼睛,一束阳光透到我的脚边:今天天气真好,又可以去跳长绳了…… 还没等我想完,忽然,一阵大风吹来,这仿佛是一群狮子在怒吼,又好像一群老虎在咆哮。窗子剧烈地摇动,仿佛下一秒就要破了似的。过了一会儿,我要上学了,刚打开楼下的大门,一阵大风括了过来,这简直就是寒风刺骨啊!于是我赶忙戴上帽子,把手塞到口袋里去。可是风还是不罢休,“嗖”地一下钻进袖子里。今天的风可冷啊!

\section{2021年11月30日 周二 晴}

孙轩睿

发试卷了!

你一定会想,竞赛卷早发了,怎么还有呢?告诉你,今天又做了一张试卷。发下来后,我算了一下前一张,扣了6分,已知第三篇阅读扣了3.5分,已扣9.5分。“加油,作文不扣分!”我喃喃着。第二张发到了,老规矩,挡住,一点一点看。先是“-”,好吧,扣分了,总不可能是“-0”吧,希望扣少一点。“1”,扣了一分,刚好89.5分,差了0.5分,就差0.5分啊!我都不知道怎么说自己了,卡得死死的。

\section{2021年11月30日 周二 阴雨}

马涵悦

大家好!欢迎来到半小时一度的“冤家”时间!

本次的出演者:郭芙蓉、吕秀才

表演开始!首先,吕秀才侧着身体趁着郭芙蓉不注意,以迅雷不及掩耳之势拿走她的橡皮壳,吕秀才偷袭成功!郭芙蓉很快地发现吕秀才拿了她的橡皮壳。于是把他手里的壳“涮”地一下抓住了。在俩人僵持不下时,李大嘴说:你们都先放手,把橡皮壳留在郭芙蓉桌上,和平共处呗。于是,郭芙蓉猛得松开了手,吕秀才…… 吕秀才没有放手!他赢了!只见吕秀才开心地笑了,扭来扭去庆祝着。郭芙蓉气得七窍生烟,把桌子往后一拉。“砰”吕秀才卡在了桌子与椅子的缝里,郭芙蓉获得了最终的胜利!由此我得出了一个结论:不作死,不会死。(NO
zuo,No die)

\section{2021年12月1日 周三 晴}

陈思涵

今天是12月的第一天,阳光很好,温柔地洒在校园里的每个角落。冬日的暖阳很美好,亮堂堂的,每个给照到的人和事物,都似乎会发光呢!距离2022年还有最后31个日夜,再过700多个小时,新的一年就到了。我的意思是,再过700多个小时,你再不努力,就要被人超越了!所以,请珍惜住最后的这一些时间,再背水一战一次,争取在2021年的小尾巴上绽放自己的光彩,为2021这一年画上一个完美的感叹号!

\section{2021年12月1日 周三 晴}

黄赫茗

今天天气是异常的冷,寒风吹着我的脸皮,像在打我的骨头,几分钟后便没有了知觉。

\section{2021年12月7日 周二 晴}

孙轩睿

唉,周五就要去装牙套了,令我又紧张又兴奋。

怎么说兴奋呢?因为戴上牙套就感觉突然长大了,而且班里很多人都戴了牙套,我也要戴上牙套感觉一下,就有了共同语言,没事时还能聊聊这个。但紧张是未免肯定会有的,毕竟你又不知道有什么感觉,别人的感觉也不太靠谱,因为有人怕痛,有人不怕痛,每个人的感觉都不一样,同学说:“嗯,还好。”“就那样。”

“难受。”医生说:“不好说,有人说难受,有人说不难受。”哎,听了这么多感受,我心里更没底了。

你说整牙到底难受不?

\section{2021年12月13日 周一 晴}

宋旻峻

今天一早,妈妈就连忙问睡眼朦眬的我:“今天是不是要听写?昨天准备了吗?”“没…… 没有。”我支支吾吾地说,“没事,反正也简单,不会错的。那就这样,错了就一年不能吃KFC!”“好的。”妈妈爽快地答应了。

今天中午听写完后,老师把每一本有错的听写发还给同学。我对自己有信心,就一直埋头做作业。
唉?老师怎么向我走来了?不要啊!我的KFC!老师的每一步都像是妈妈“审判”的脚步声,沉重而可怕。不过还好,是我的同桌有错,并不是我。真是虚惊一场。

\section{2021年12月14日 周二 阴}

杨若君

前几天,我们家买了一个柚子,它的皮都黑一块黄一块的,十分丑,看起来都坏了,一点儿香味儿都没有。今天回家我实在是想吃点儿水果。爸爸说:“那有个柚子,你自己去切吧。”我一脸不情怨,想:这肯定是个坏的,切了也没用。可就当我刚切开时,一股清香流到鼻子里,我一尝,啊!这太甜了吧,我还以为是坏的,没想到这么好吃。看来决定一个人的不是它的外表,而是他的内在啊!

\section{2021年12月14日 周二 阴}

孙轩睿

今天下午第三节,我们长绳一队和二队的队员一起去了篮球场比赛。哨声响起,一队又一队开始跳,我们一队的队员一个接一个进入长绳,跳完就迅速走。毕竟是比赛,谁都没有松懈,谁都不敢放松,一个个注意力十分集中,比平时高子200\%,所以我们一次也没有断,而且速度较快,超越自己的纪录,跳了373个;反观二队,因为一队少人,而一直被挖人,配合不太好,只跳了250多。

平时还要更努力啊。

\section{2021年12月15日 周三 晴}

陈姿伊

今天下课的时候,孔俞澄偷偷地打了宋旻峻一下,宋旻峻并不知道是谁打的他,于是生气地说:“是谁打的我?”孔俞澄趁我在写作业,偷偷地指了指我。然后宋旻峻便转向我,弹了一下我的脑门还一边说:“叫你打我。”我感到又迷茫又憋屈:“你打我干嘛?”“刚才不是你打我的吗?别装了。”我疑惑至极:“刚才我在写作业,我没打你。”接着宋旻峻仿佛知道了什么,瞪大了睛睛,缓缓转过头去,看了看正在一边坏笑,一边扭来扭去的孔俞澄。我和宋旻峻都明白了,原来就是他搞的恶作剧!

\section{2021年12月15日 周三 阴}

陈思涵

眼看这一个学期又快要结束了,马上又要到期末考了,可真快啊!最后一点时间,要好好努力了,把不好的变好,把好的变更好,不能到期末考时临时抱佛脚,那都没用了!所以,同学们,我们要借这最后一点时间补上那些漏洞,争取交出一张完美的答卷!

\section{2021年12月7日 周二 晴}

马涵悦

欢迎大家来到“暴力女”专揖之——郭芙蓉!

出演时间:电脑课下课 地点:走廊。

PK人物:吕秀才,郭芙蓉 掌声有请!

首先,吕秀才趁郭芙蓉不注意,刷一下子来了个“摸头杀”。郭芙蓉灵敏转身,踹了吕秀才一脚,同时两人异口同声地“哇”假哭起来,(路人甲:???路人乙:呐尼?)郭芙蓉又把吕秀才的书“啪”抢了过来,吕秀才也飞快地拽住郭芙蓉的头发,一脸傲骄地说:把书还给我!郭芙蓉无奈,只好把书给吕秀才。只听啪一声,你猜怎么着?书打到吕秀才脸上了!(路人甲、乙、丙、丁:哈哈哈哈)没等吕秀才反应过来,郭芙蓉早已逃之夭夭了。

\section{2021年12月14日 周二 阴}

马涵悦

哈罗大家好!又见面了!今天我们把“暴力女”先放一放,咱来讲讲耍衰男白展堂之——滑铲!!!

出演时间:下课扫地时

出演者:白展堂、郭芙蓉

地点:教室、走廊

大家掌声欢迎!!!只见白展堂飞速加跑,中途半跪下身,说了句,滑铲……。接下来咱来看看他如何把这招用到暴力女当中。白展堂先是照老样子给郭芙蓉一个大大的摸头杀(呀!还把她发型弄乱了),然后飞速逃跑,郭芙蓉立马火冒三丈。郭芙蓉追到一半,拉开了足够距离的白展堂突然转身,就向郭芙蓉跑来。这下子把郭芙蓉整懵了,愣在那儿一动不动,随着咱们熟悉的一声:“滑铲\textasciitilde 起飞!”白展堂正巧滑到了郭芙蓉前边。正巧后面有人撞了郭芙蓉一下,好巧不巧吗,郭芙蓉一脚踩到了白展堂的手。伴随着“啊”的一声杀猪叫,郭芙蓉(杀猪刀)竟意外的“杀猪成功”!!!(请原谅本作者已笑晕在厕所:鹅鹅鹅鹅——)

本剧终!

预告:下集将回归暴力女主题,敬请期待!

\section{2021年12月20日 周一 晴}

黄赫茗

我现在发现,不仅手机上抢红包要手速,考试时也要手速。因为考试的时间有长有短。遇到短时间的考试时,如果手速不快,那么可能你的作文还没写完,卷子就被收上去了。还有些同学就难过了:“我明明会写,可就莫过于手速太慢……”还有一种情况,你的卷子加上作文有98分,可你的卷面很差,老师给你扣了10分,现在就成了88分,这就不能怪别人了啦!

\section{2021年12月21日 周二 阴}

马涵悦

大家好!本次“暴力女专辑回归!!主题——冤家同桌!

出演时间:(依旧是)电脑课下课。

出演者:吕秀才,郭芙蓉

出演地点:走廊,教室。热烈欢迎!!!

电脑课下课后,一群女生听了莫小贝的话后哈哈大笑,还望着吕秀才。(路人甲、乙:神马玩意???)原来郭芙蓉和莫小贝上电脑课的时候一直用语音输入:吕秀才是个大美女…… 一场大战一触即发。回到教室,吕秀才拿起笔就怒气冲冲地质问郭芙蓉:“你干什么”(尽管他还是笑了)郭芙蓉也火了,(尽管她也不解谱地笑了)直接一手捏住了吕秀才的脸,眼睛里好像要喷出火来。这时的吕秀才可怜,弱小又无助啊!吕秀才:唉,我太难了!!(现众:额鹅鹅鹅鹅鹅鹅\textasciitilde\textasciitilde)郭芙蓉狠狠滴捏了一会吕秀才的脸,终于放手了。吕秀才委屈地揉着自己的脸,用手指着郭芙蓉,可怜巴巴地说:“她打我!”装出一副被欺负的样子,郭芙蓉刚把头转过去,听到这话立马又把头转了回来,这回直接用两手夹住吕秀才的脸,咬牙切齿道:“我打你了吗?打你了吗??”啊!!吕秀才不甘示弱的用头顶郭芙蓉,郭芙蓉立马火了,动了直格,用手一只捏他的脸,一只拽他头发。“啊\textasciitilde\textasciitilde\textasciitilde{}”的惨叫声响彻云霄,画面不忍直视啊!啧啧啧…… 年轻人,还是别招惹暴力女啊!否则不听老人言,吃亏在眼前!

本剧终!

无奖竞猜:下次内容是什么呢?让我们期待吧!

\section{2021年12月22日 星期三 晴}

陈姿伊

放学回家的路上,有几只猫突然出现眼前。我一眼就看到了一只浑身雪白的猫,拥有着美丽的异瞳:左眼橙黄,右眼碧蓝,尽显高贵。一只拥有着硕大身躯的猫,一个健步冲了上来,一口咬走我扔在地上的小饼干;一只瘦削的猫被挤在了猫群之外,抢不到一点儿食物,我只好为它开小灶,单独给它投食,却还是被一旁的“大胖猫”抢走了。

\section{2021年12月22日 周三 晴}

陈思涵

今天放学后在小区里走在回家的路上,我前面走着一位阿姨,约莫40来岁的样子,手提着一个菜篮子,一瘸一拐地向前走着。看到了这幕,我的心里波澜起伏:“我应不应该走到她前面呢?要是走到她前面,她要是看到我了,心里一定会很难受的吧!哎……”我脚步一顿,转身走了……

\section{2021年12月23日 星期四}

孔俞澄

今天,我一身汗,头上都是头皮屑,妈妈叫我去洗个澡,我不情不愿自地开了厕所里的花洒,于是我哼着小曲,悠哉悠哉地洗了起来,过了许久发现脚底热乎乎的,低头一看发现洗澡的地方积起了水,不时还往外漏,我大惊失色,赶紧换上衣服清理了积水。

\section{2021年12月23日 周四 晴}

盛夏颖

大家好!今天依旧没有“暴力女”专辑,今天的主题是——冤家同桌!!

出演时间:数学课上出演人物:吕秀才、郭芙蓉、李大嘴、章老师

出演地点:教室内主题:冤家同桌大作战之被老师抓现形

掌声欢迎!!(如果得罪了各位,对不起!!别伤害我哈哈!)

第二课是数学,我正仔细听着章老师讲解着重难题,可前方却传来一阵阵声音,我向前一看,发现吕秀才作死般的用水笔向郭芙蓉的笔盒凑去,我却风平浪静(因为他平时没少搞我)。郭芙蓉听见李大嘴在“嘻嘻”地笑,立马意识到不对(被吕秀才搞习惯了),飞速拖转过头,看见了开头的那一幕。她用她的手腕向着吕秀才的手腕“砍”去!我忍不住感叹一句:“omg的!厉害啊!”(导演:搞没搞错,看看剧本好不好!)眼尖的章老师发现了吕秀才的action(小动作)。一边叫着:“吕秀才!”一边赶忙走到他旁边用力拍了一下吕秀才的背。吕秀才背上便有了粉笔灰……

完结!!撒花!!

\section{2021年12月28日 周二 晴}

孙轩睿

昨天又发试卷!

不知为何,我的2张卷子“分离”了,一起交的,回来就成“两半”了,先发到了后一张(其实后一张只是半篇阅读)我先算了一下,扣了5分!内心崩溃的,前一张有2篇阅读,估计要凉,上不了90分。我心中立下flag:如果上90分,我就…… 还没立完前一张来了,先看一下再说。第一面全对!后面两阅读错1题,扣1分,共扣6分,94分。正好flag没立完,我总不能像傻子一样自我惩罚吧,于是我继续默念:上90分,我就…… 喝一口水!(我承认耍赖了)然后来了一口水。

第一张试卷来得好及时!

\section{2021年12月28日 周二 阴}

马涵悦

哈罗!“暴力女”专楫又来了!!今日主题依旧是冤家同桌。

出演时间:放学前。 出演人物:郭芙蓉、白展堂、吕秀才

出演地点:教室里。 主题:冤家同桌

要放学了,大家都在理书包。只见吕秀才趁乱作死的一脚踩在郭芙蓉书包上,还在上边用力的地摩擦了几下。(啧啧啧,这胆子不要太大发)正好郭芙蓉从徐老师那儿过好关回来,看见这情景,立马火冒三丈,一脚踹向吕秀才。吕秀才几个灵魂走位,哎嘿!溜了。郭芙蓉破口大骂:我+x-÷@……《+{]}十一@……!正好看到一旁的白展堂在弯腰整理书包,一手直接把白展堂的头向下一摁,“啊!郭芙蓉你干吗?”白展堂揉了揉自己的脑袋,一脸懵。郭芙蓉:“你踩我书包干啥?”白展堂:“我没踩你书包啊?”郭芙蓉:“那书包上的印子谁踩的呀?啊?”白展堂十分委屈,指着书包说:“那是吕秀才踩的呀?关我啥事?”郭芙蓉一身怒火无处发泄,于是一把捏住了白展堂的脸,咬牙切齿地说:“吕秀才是你兄弟,那他的债就由你来偿!!!”(白展堂内心:呐尼?吕秀才你个猪队友!啊!)郭芙蓉又狠狠踹了白展堂几脚。白展堂惨叫连连,郭芙蓉总算放过了他。白展堂刚弯下腰来想赶紧理好书包逃离郭芙蓉的魔爪时,郭芙蓉一个大脚踹在白展堂的背上,白展堂“喔”了一声直接趴在了书包上,久久起不来。正当白展堂想爬起来时,李大嘴拿着作业本冲了过去,白展堂露在过道上的头被重击了一下,呃…… 好像…… 打脸了?!(观众甲、乙、丙、丁:“鹅鹅鹅哈哈哈……”本作者已笑晕在座位,请谅解。

本剧终!撒花!

让我们欢迎许智涵主编!我这主编就先走了,886!下周不见不散!!!

\section{2021年12月29日 星期三 晴}

陈姿伊

今天回到家,我就发现了不对:狗为什么没来迎接我呢?平常它可是又蹦又跳地扑向我呀!我一眼望去,客厅里没有狗,厨房、书房和卧室没有它,就连它经常睡的厕所都没有它的身影。我的心跳直接漏了一拍,狗不会出什么意外了吧?就在我要跑去告诉家长时,我看到了一双反着白光的大黑圆眼:它居然趴在我的客桌下!唉,这家伙可真不让人省心呀!

\section{2021年12月29日 周三 晴}

陈思涵

转眼间又是一年,还有2天2021年就要过去了,一个崭新的2022年又要到了。剩下的48个小时,是选择自由,放松一下,还是选择奋斗,再坚持下。但请保持热爱,继续奔赴下一场山海。新的一年新的开始,希望大家都可以整理下自己的容面,用一个崭新的自己来面对。

\section{2021年12月30日 星期四 晴}

孔俞澄

一个月黑风高的夜晚,我去学校里拿作业。走进教学楼周围一片死寂,一根针掉到地都能听见。我飞快奔向教室,只听“答”、“答”,我从黑暗中隐隐约约看到前面的灯不停闪烁。突然一个黑影飞快地从那边闪过。我的心紧张地跳动起来了,不会是鬼吧!不,怎可能有?于是我飞奔上楼,三步作两步,终于跑到了教室,飞快地拿了我的作业,跑下了楼。下次整理时可得注意一点。
