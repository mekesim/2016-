\chapter{第二期}

\section{2019年4月24日 星期三 小雨}

徐博悦

昨天我准备做作业的时候,发现一只大蚊子在我的书上。我用力一拍没拍中,我又拍了一下任然没拍中。突然,蚊子飞到我的右脸上,我用力一拍,``啪!''结果没拍到蚊子却打了自己一个耳光。我火冒三丈,拿起杀虫剂就对着它喷。但它还在飞,我这才发现杀虫剂过期了,对蚊子没什么用处,呜呜\ldots{}\ldots{}我好倒霉啊!

\section{2019年5月9日 星期四 晴}

陈思涵

五一放假期间,我来到了黄山风景区爬黄山。我一手拄着登山杖,一手叉着腰,一步一步小心地走着,生怕把脚给扭了。经过一个上午的挥洒汗水,我终于爬上了光明顶,当时,我特别害怕,怕自己会摔下去,但妈妈一直鼓励着我,让我不要再害怕了,抬起头,坚定地走下去。我听了妈妈的话,顿时有了信心。在下山的路上,我为中华感到自豪,因为我身在这奇峰耸立的黄山之中啊!下了山,我终于知道了什么事坚持,坚持是在你想放弃的时候却继续坚定地走下去,坚持是你迈开脚步的那一瞬间!

\section{2019年5月16日 星期四 晴}

陈思涵

吃完晚饭,妈妈拿来了一盒果泥,准备给我当饭后小吃。谁知,被小宝看见了,他一看见果泥,两眼就一直盯着果泥,怎么也转移不了他那炯炯的目光。小宝这时才恍然大悟:应该上前抢!他立马往妈妈站的地方奔去,妈妈手往高处挪,小宝跳起来,拿到了他盼望已久的美食。妈妈岂能善罢甘休,她问:``小宝,那你喜欢我吗?''小宝一刻也不迟钝,立刻回答``喜欢我!''站在一旁的我扑哧一声,忍不住笑了。我想,这小宝可真搞笑。

\section{2019年5月20日 星期一 阴}

张欣瑜

春天来了,春姑娘早早地给柳树换上了新衣。你看,西湖边的柳树正享受着太阳公公的温暖,显得可精神了!远远望去,千万缕的柳枝在春风中尽情地摆动着柔软的身子,多像一个美丽的姑娘在抖动她刚洗过的长长的发辫儿啊!这个情景让我想起了《咏柳》这首诗:碧玉妆成一树高,万条垂下绿丝绦。不知细叶谁裁出,二月春风似剪刀。春天的柳树真美啊!

\section{2019年5月20日 星期一 阴}

宋旻峻

今天下午,老师让我们到外面看书。看着看着,忽然冒出了一只头:``借我一本书''我吓得魂飞魄散。定睛一看,原来是同学向我要书。于是,我给了她一本书。但我想开心一下,就把另一本也给了她,说:``买一送一啊!''逗得大家哈哈大笑。

\section{2019年5月21日 星期二 晴}

杨航熙

今天我因为实在太饿了,所以我偷偷的把一包山核桃塞进了书柜里。''山核桃又大又甜,吃完一包我肯定还要再吃一包。''我想。果然,我吃了一颗就有一种想飞的感觉,口水流得满衣服都是,脏兮兮的,有洁癖的人大概看了会吐。我三口两口的把那包山核桃吃,不,是吞了下去。我想又一次''窃吃''成功了!

\section{2019年5月22日 星期三 晴}

许智涵

在天街吃完晚饭回家的时候,我正看着窗外的景色,忽然一阵睡意向我袭来,我竟靠在后排座椅上昏昏沉沉地睡着了。我恍惚间觉得汽车停了下来,接着马上又转弯,后来又下坡\ldots{}\ldots{}我觉得眼前明亮了许多,原来是到车库了,我猛然清醒,赶紧背上书包,打开车门走了出去。唉!看来以后要早点睡觉,睡眠一定要充足呀!

\section{2019年5月23日 星期四 晴}

陈奕铖

今天,我拿起乒乓球拍,自己一个人在墙壁上打乒乓球。小小的乒乓球,在球拍上来回地跳着,像一只活泼的可爱的,小松鼠在森林里跳动着。我拍了一分钟不到,却断了十几次,我垂头丧气的时候。我妈妈走过来说:``你拿的球拍手法不对,妈妈示范一下。''我看了看发现拿球拍是大拇指和食指在球拍上,后面中指、无名指和小指在下面顶着。怪不得妈妈拍球比我历害。手法错了,怎么也拍不好球。我学了老半天,终于学会了一点点,我高兴地笑了。我想只要我要努力多练,一定会学好的!

\section{2019年5月27日 星期一 雨}

余蕙琳

今天去医院看好病,准备返回学校时,发现爸爸没有来,原来爸爸还在排队配药。于是我和外婆开始了等待之旅。我一会儿站一会儿坐,十分不安稳,外婆左盼右盼似乎在寻找爸爸。时间一分一秒过去了,太阳照在我们的脸上火辣辣的,我无聊地开始数蚂蚁了``一、二、三\ldots{}\ldots{}.'',``滴答滴答''呀,头发上,一颗颗像米粒一样大的汗珠从我发尖掉落。终于,爸爸回来了。

\section{2019年5月28日}

付梦欣

``
妈妈,爸爸今晚回来吃饭吗?''我问道,``不知道,你自己打电话问吧。''妈妈不紧不慢地说。我长叹了一口气,哎!妈妈每次这样回答,爸爸十有八九肯定是不回来吃饭。我多么希望爸爸能陪我一起吃饭,陪我一起写作业,陪我一起嬉戏打闹\ldots{}\ldots{}哪怕能和爸爸坐在一起大眼瞪小眼也行啊。想到这里,我的眼泪就像断了线的珍珠似的从脸上滑落下来。我多想告诉他:``我需要您的陪伴,需要您抽更多的时间留给我,陪我一起成长。''

\section{2019年5月27日 星期一 阴}

张欣瑜

清晨的云,真是千姿百态。它们有的像一个个大型巧克力冰激凌,让人看了忍不住想咬上一口;有的像棉花糖看上去软绵绵的;有的像一个完整无暇的爱心,令人想摘下来美美的欣赏一番并把它收藏起来;有的薄如轻纱,犹如仙子的衣带飘浮在空中;有的像羽毛,轻轻地飘浮空中;有的像一座挺立雪山;有的像一匹骏马奔驰在辽阔的草原上\ldots{}\ldots{}云那千姿百态的形状,让我陶醉。我爱清晨的云。

\section{2019年6月4日}

付梦欣

放学回家的路上,我一边走一边回想学校里的趣事。突然,一个不明之物``嘭''的一声掉在了我的脑门上,又清脆地掉到了地上。``这是什么鬼东西啊!''我尖叫道。抬头望去,交叉的树枝和稠密的树叶遮住了太阳,李子一颗挨着一颗躲在树叶下乘凉。近看,李子就像一颗颗闪闪发光的红宝石,点缀着李树。远看,沉甸甸的``红宝石''把树枝压弯了腰,好像在向路人们鞠躬问好,炫耀它的硕果。

\section{2019年6月3日 星期一 晴}

余蕙琳

今天放学和杨若君回家时,遇见了我们班的戏精:李恩博。为什么说他是戏精呢?看了这篇文章你就知道。这会儿他从书包里拿出水杯,像孙悟空一样抡起水杯就向杨若君打。``呀!救命呀!''杨若君慌忙而逃。可现在他又把书包挂到手腕上,装出书包很重的样子。嘴里还喊着``重死了,重死了!''可他还蹦嘣跳跳地走着。走到半路,他不再往前走,而是一个180度急转弯向另一个方向去了,还像豹子一样冲向马路,他不说书包重吗?怎么又变成这个样了。哎,真是个戏精。

\section{2019年6月3日 星期一 雨}

为了看杭州落日的绮丽景色,我吃完饭就来到江边。这时候,太阳已经西斜了,它像个害羞的姑娘露出那红彤彤的脸蛋,周围绕着一圈光环。那天空是淡蓝色的,像明净的湖水,现在越来越蓝,越来越浓了,简直有点儿红了呢!落日还像喝了酒的红脸醉汉,跌落在山那边,江边连绵起伏的群山在夕阳的照射下像披上了闪闪发光的金衣。这时候,落日把水和天映得一半通红,一半金黄,看起来分外壮观。我情不自禁地赞叹:``啊!杭州的落日可真迷人!''

\section{2019年5月27日 星期一 阴}

叶雨飞

晚上,我正做着作业,忽然感到非常困。我努力克制住自己,不让自己睡着。但最后我抵挡不住睡魔对我的攻击,躺在椅子上睡着了。美梦做到一半的时候,我忽然听见妈妈大喝一声:``叶雨飞,这都什么时候了?你还有没有时间观念啊!''我一下子被惊醒,看了一下时间,天哪!已经九点了!我慌忙拿起作业本,以最快的速度写起来,到九点半才写完。看来以后得早睡早起,否则后果会非常严重!

\section{2019年6月3日 星期一 晴}

叶雨飞

今天下午,殷教练让我和史佳卉双打严秋实和李卓涵。这也太不公平了!我们俩都在想,第一,他们是男孩,而我们是女孩;第二,他们全是09年的,而我们全是10年的。殷教练又说:``男生如果输了,跑两组步伐!''我们松了一口气。比赛开始了!我们心里都想着,千万要赢啊!前半局,我和史佳卉配合得相当好,领先了6分。后来,越战越勇,你一分我一分,最后大胜。嘿嘿,我们兴高采烈地看着他们跑步,他俩则愁眉苦脸的。

\section{2019年6月10日 星期一 暴雨}

叶雨飞

今天下午,我们正在订正数学作业,忽然,我听见一阵哗啦啦的声音,扭过头一看,只见外面下着瓢泼大雨(或者说倾盆大雨),未很惊奇:刚才不是还晴天吗?看来老天爷变脸也很快啊!我心想:糟糕,待会儿去练羽毛球,不会被淋成落汤鸡吧?但马上又想起来今天值日,希望值日值晚点,说不定雨就停了!后来,等我们值日完,天又晴了。还好,没被淋成落汤鸡!

\section{2019年6月13日 星期四 雨}

陈奕铖

今天早上,叮铃铃一阵闹钟声把我在睡梦中惊醒。原来是我的老朋友闹闹在叫我起床,我伸了伸懒腰。说:``闹闹你别吵了,现在是6点,我的鸡腿还没吃完呢!''闹闹一直叫``叮铃铃、叮铃铃\ldots{}\ldots{}''我马上起来把闹闹拿到床上,迷迷乎乎的一边关闹闹,一边说:``闹闹还我鸡腿\ldots{}\ldots{},''

\section{后记}

同学们从今天开始就开始放暑假了,大家是不是很激动呢?

过去的一个月,在詹老师,各位同学和各位家长的努力下,《闲言碎语》活动继续坚持着。詹老师每天要花大量的时间批阅每位同学们的笔记,感谢老师的默默付出。同学们更是辛苦,在完成作业和各种培训班的空余时间还要绞尽脑汁一篇日记,确实不容易。家长们也积极响应,百忙之中抽时间把孩子的文章打印成电子稿发过来。总之,在大家的共同努力下,有了我们的闲言碎语第二期。

闲言碎语的初衷是鼓励大家能坚持写作,提高写作水平。写作是一项非常有价值的技能,遗憾的是我上学时经常语文不及格,写作水平不好,常为此而遗憾。前几天听吕丽娟老师的讲座说:“执笔人有历史解释权”。意思是说,后人眼中的事实,是由前人笔下的文章所描绘的。罗贯中在《三国演义》中把曹操写成一个十恶不赦的大坏蛋,《三国演义》的影响力实在是太大了,所以大家都认为曹操是个奸诈多疑的小人。可是如果读正史《三国策》中的记载,大家会发现曹操其实是一个胸怀宽广,能文能武的大英雄。要不然,怎么会有那么多优秀的人投靠曹操呢?

所以,能写好文章是一件神圣的事情。如果读到一篇与自己想法一致的文章,作者写出了你想要说的话,那是多么地畅快。希望大家能体会到写作的重要性,不仅仅是为了应付考试,更是为了能更好的表达自己。

在公众号发布的每日之最,只是为了鼓励同学们的积极性,同时希望通过这样一个仪式,鞭策大家能一起坚持下去。没有选中的同学切莫灰心,家长也切勿打消他们的积极性。此时正值芬兰的春夏之交,路边长满了各种野花草。前几日遍地金黄的蒲公英铺洒在绿色的草地上,在绿色的映衬下黄得那么耀眼。没过几日蒲公英全白了,不知不觉间又冒出斑斑点点的,说不上名字的白花、黄花点缀在更茂盛的草丛中。也许在人们看来金黄的蒲公英是最漂亮的。但是在花花草草看来,他们都在享受着阳光的温暖,竭力地展现自己最美好的时光,结下丰硕的种子。

另外,想跟大家分享写作方法。我不是专家,所以不敢胡言乱语,且引用林语堂先生关于写作的建议:第一,写作如作画,可分为临帖和写生。临帖就是要多看书,模仿前辈的方法进行写作。本期中一位同学描写自然景物的文字,观察细致入微,文字生动优美。定是她平时积累的写作素材比较多。写生就是自己创作了。第二,初级阶段可以以描写文和叙事文为主。林语堂先生认为因为阅历比较少,刚开始比较难以写出议论文。描写文和叙事文写好了以后,写议论文,也就功到自然成了。本期中大多数同学写的就是叙事文,其中有位同学讲述了家中的一件小事,她妈妈告诉我,她在写这篇短文的时候把自己感动得眼泪吧嗒吧嗒直落,旁人读了,也深为感动。请相信,能感动自己的事,也一定能感动他人。所以,写出心中所想,记下感动自己的事,也一定能感动读者。

最后,祝大家暑期愉快,四年级再见。
