\chapter{第十期}

\section{2022年2月28日 星期一 多云}

张欣瑜

一支鲜嫩欲滴,灿烂如朝霞的红玫瑰静静的躺在我的桌前,我入神地凝望着它,它的颜色,它的姿态,它的芬芳吸引着我,我轻轻地把它放在我的手中。它是那么美,红得似火的花瓣,一层一层的环抱着,好似一个小姑娘灿烂的笑脸,又好似一只盛满红酒的高脚杯。我凑到它面前闻了闻,一股淡淡的清香飘进了我鼻子里。

\section{2022年3月1日 星期二 多云转小雨}

丁凝

放学路上,我看见一只鹊鸲在一大片衰草中蹦跳,我不禁站得远远的观察它。过了不久,一只猫悄无声息的走进了我的视野,他慢慢的往前挪动,几乎走一步就停五秒,并且压低了身体,不让鹊鸲发现他。我聚精会神地盯着他们,不敢出声。毫不知情的鹊鸲继续蹦跳着,没有丝毫的防备。它经常低头啄啄草,15米、12米,10米、9米,猫发现鹊鸲,进入它的攻击范围了,一双眼紧紧的盯着,盯着鹊鸲,猛然间,猫以迅雷不及掩耳之势冲了出去,鹊鸲刚反应过来,往前跳了几下,接着翅膀张开,就被咬住了翅膀,飞不起来了。我连忙赶上去,追上猫,阻止这场弱肉强食的发生。猫扔下了鸟,逃走了。我把鸟交给了旁边,同样观战的一个老人,这才离开。

\section{2022年3月1日 星期二 晴}

杨若君

今天我们在上体育课的时候,我在不经意之间,发现春天好像悄无声息地来到了我们的身边。小草们变成了嫩绿色的,随着微风轻轻佛动。枝头,小芽探出了一个小头。阳光温暖地洒在校园的每一个角落,万物都开始生长。这时,虽然没有在空中飞舞的蝴蝶,但是有在枝头歌唱的小鸟。我感受着阳光带来的这份温暖,感受着微风带来的这份清凉,感受着鸟儿带来的这份愉快。而这一切都说明了春天来了!

\section{2022年3月1日 周二 晴}

陈思涵

转眼间,2022年又是2个月过去了,今天是3月的第一天了。时间真的是转瞬即逝啊!我都还在回味春节,这新的一年就过去了。想到这,我不禁感慨:“这时间过得真快啊!再过3个多月,我就要毕业了呢!”真是“岁月不待人”啊!我们应该抓住每一分每一秒,最大程度地利用时间!

\section{2022年3月8日 周三 晴}

陈思涵

今天是个特殊的日子,没错,就是“3.8女神节”!我也给妈妈准备了一个小小的礼物,当作一个“惊喜”送给了她。其实,在过去的一段时间,许多女子也是在各种各样的赛场上大放光彩。比如在东京奥运会上夺得首金的杨倩,或是全面发展的“天才”女孩谷爱凌…… 她们的出现,让更多人看到中国女子灿烂的一面,就像中国女足一般的铿锵玫瑰还有很多,谁看了不说一句“你可以永远相信中国女子”呢!

\section{2022年3月15日 周三 晴}

陈思涵

周末的时候,我练完琴想趁着好天气下去散散步。可就在门要关上的那一刻,弟弟整个人扑了上来,硬拉着我让我带他一起下去(我后来才知道他以为我下去买吃的,其实是来蹭吃的),我无情的拒绝了他,谁知他“哇”的一声就哭了出来,我顿时愣住了,不过下一秒就跑到妈妈面前,使出了大招——我也学着弟弟的模样,两行眼泪流出眼眶,在妈妈面前哀求,可妈妈却不领情,甩开了我的手:“去一边去!大戏精,送你去学表演得了!”我也只好带弟弟,不过幸好天公不作美——下雨了!我拉着弟弟的手就往回跑,心里不断欢呼:“终于甩掉这个“重任”啦!“

\section{2022年3月22日 周三 小雨转阴}

陈思涵

周末的时候,妈妈说要给我做鸡米花了,看着她匆匆下楼的身影,我不禁有些疑惑:妈妈居然破天荒要做鸡米花?不可能!估计不好吃!想到这儿,我也就去做别的事了,早就把这事忘在了脑后。一个下午妈妈都进进出出,忙得不得了。晚饭时,看着桌子上一碗金黄流油的鸡米花,我惊讶地望向妈妈,夹起一个往嘴里一咬,外面脆脆的,里面软软的,好吃极了!看来是我低估妈妈了。

\section{2022年3月2日 星期三 晴}

戴瑞彤

冬奥会已经正式落下了帷幕,但冬奥会的顶流冰墩墩却还是那么抢手,这不,昨天就有人带了个冰墩墩摆件来。他是早上晨读的时候拿出来的,他一开始只是跟周围的同学说了一下,但渐渐地,一传十、十传二十…… 全班同学几乎都知道了这件事。有的同学坐不住了,直接放下了手中的书,小跑过去,只为看一眼冰墩墩的样子;有的同学虽然没过去看冰墩墩,但魂也早已被钩走了。几分钟后,整齐的读书声才再次从班里传出来。

\section{2022年3月2日 周三 晴}

姚琪

哈啰!今天没有“暴力女”,但有一个“杂技艺人”!你们猜是谁?哈哈,当然是我前桌的米小圈大哥咯!

出演时间:数学课。 出演地点:教室内。

出演人物:章老师、米小圈

主题:杂耍艺人

数学课上,我们正在学“如何算圆柱的表面积”,前桌米小圈拿起一块橡皮,开始了他的表演。只见他用右手把橡皮抛起,再用左手接住。这个动作作为基础,而后越抛越高,最后趋于一个圆。米小圈不亦乐乎,认真的不得了,(老师:你在学习上有这股劲就好了!)章老师突然转过来了,“眼观六路,耳听八方”的米小圈早有准备,收手坐端正,全然不顾以“百米冲刺”速度下落的橡皮,章老师看了一定很无语,叉着腰盯着米小圈,过了好久才转移视线。等到“完全安全”了,米小圈快速捡回橡皮,继续他的杂耍生涯……。哎!在下佩服米小圈的勇气与“智慧”!
本剧终……

敬请期待下一期许智涵的来稿!下周再见!拜拜!

\section{2022年3月7日 星期一 雨}

张欣瑜

春天悄悄地来临了,早晨,一阵微风吹过,风中还夹着一丝丝清香,纤细柔嫩的迎春花瓣上,多了几颗晶莹秀亮的小露珠。露珠闪闪发光,它们凝住了迎春花淡淡的清香味,在花上滚来滚去,在晨雾的衬托下,显得既宁静又不失活泼和生气。丝丝春雨,悄无声息地落下来,滋润着伸展腰肢的迎春花,那黄澄澄的小脸上衬托出文静、清幽。

\section{2022年3月8日 星期二 阴}

丁凝

小区里的木兰花终于盛开了,白灿灿的一大片,犹如三月下雪,可远远的便闻见浓郁的花香,沁人心脾,招来一大片的蜜蜂蝴蝶,小孩子们纷纷拾起落在地上的花瓣,相互比较,谁的最大最美。仿佛要跟着凑热闹似的,四周桃花也开了,杨柳也绿了,一派生机勃勃的景象。真是应了那一句,草树知春不久归,百般红紫斗芳菲啊!

\section{2022年3月14日 星期一 雨}

张欣瑜

玉兰树高大而挺拔,树上长满枝丫。一朵朵雪白的玉兰花将树梢压得很弯,就好似一团团雪架在树上。花瓣是洁白的,只有在靠近花蒂处才有一丝粉色。洁白的花上不染一丝污垢,可以与荷花的“出淤泥而不染”相提并论。

\section{2022年3月14日 星期二 晴}

陈姿伊

今天放学回家时,我和外婆才发现上楼的卡忘记带了。没有卡可不行呀!那是上电梯必须要用的东西。外面下着倾盆大雨,这天气,既使想找人帮忙恐怕也找不到。最终我提出的方案终于通过了外婆的审核自-我先从楼梯爬上去,进家里把卡拿出来,然后再乘电梯把在下面等我的外婆接上来。

我气喘喘吁吁地爬了十八层楼之后,终于在家找到了钥匙!就当我自豪地准备下楼时,外婆居然搭乘着外卖小哥的卡上来了。唉,我真是白忙活一场啊!

\section{2022年3月15日 星期二 晴}

杨若君

以前,在我的眼里,黑夜是可怕的,令人害怕的,每当夜晚来临,我就会回想起电影里可怕的怪兽。而就在昨天晚上,我静静地躺在床上,看着窗外那黑漆漆的夜空,一个又大又圆又亮的月亮,在夜空中脱颖而出,仿佛正在静静地凝视我。这时候什么声音都没有,似乎时间已经停止了,我屏住呼吸,恨不得一片羽毛落在地上的声音都能听得一清二楚。突然,一阵蟋蟀声清楚地传入了我的耳朵…… 这真是一个甜美而又幽静的夜晚啊!

什么事物和事情都有两面性,而我们应该积极乐观地看向最好的一面,感受世间万物的美好与和谐。

\section{2022年3月15日 周二 晴}

马晨轩

《学校日常》系列\#2——内卷

(杨航熙主编的上上一期也是这个内容!)

“内卷”是一个固有名词,很多人都会说这个词,但并不知这个词的意思,我便不好意思地帮你们查一下吧!

“内卷”原本指一类文化在到达极致后,变的越来越复杂。而现在变成了网络用词,指人与人之间的互相竞争十分激烈。也指提前把工作、作业做完。

现在班上的卷王越来越多,甚至还出现了一个“龙卷风”小组,没有错,就是我们小组。

现在大家不仅内卷作业,还在内卷课外试卷。大多数人也开始喜欢上了内卷,因为我们马上要小升初了,更应该好好学习!快乐内卷!

(内卷系列到此结束了!)

(溜!了!886!有请姚琪主编进行稿件编写,大家期待一下吧!)

\section{2022年3月18日 周五 阴天}

黄赫茗

中国竟然和美国和好了!真是个难以置信的结果。要知道,自从抗美援朝以后,中国和美国的关系一直都不好(除了乒乓球)。而且去年美国还和其他国家(准确的说是被美国引诱去的)一起搞台独。今年美国的发言人却说:“我们会尽力控制那些不法分子。”

不管怎么样,这场“战役”终于结束了。

\section{2022年3月21日 星期一 雨转多云}

丁凝

夜晚是寂静的,神秘的。太阳落下,夜幕降临,月亮升起。

银白色的月光照耀大地。白天,躲藏的动物纷纷出来了,蝙蝠无声地拍打着翅膀,猫潜伏在草丛中,眼睛中闪着碧绿的光。一切陷入黑暗,一切也都显得诡异又有趣。原本白天熟悉的不能再熟悉的东西,随时都有可能给人惊吓。

\section{2022年3月21 星期一 雨}

李恩博

今天考试卷签名时,我和老爸开始讨论第一篇阅读的不科学。例如材料一说杀死活病毒,但是后面题目写的是细菌。病毒和细菌都是微生物,病毒是由蛋白质、RNA、核酸组成,而细菌比较复杂,如草履虫由RNA和DNA等组成,所以病毒和细菌不一样,所以题目是错的。第二小题题目说食用酒精可有效消灭新冠病毒,虽然75\%的医用酒精可以杀死病毒,但是血液中流着酒精才可以杀死新冠病毒,当然茅台酒也可以,虽然要喝一万多缸,但是题目也没说喝一瓶呀!而且新冠病毒会在肺部复制,你喝下75\%医用酒精,也仅仅能消灭几百个罢了,要想全部消灭,人要喝挂掉,人挂掉了病毒也会挂掉。所以考试卷的第一篇阅读,不科学、不严谨、不准确。材料一、二,题目都不严谨,说明文不能不科学。

\section{2022年3月21日 星期一 雨}

张欣瑜

十里飘香,春暖花开,你看,那如云一样灿烂的樱花,在枝头摇摇晃晃,好似害羞了一般,红中带着粉,粉中带着白,那美丽的脸庞好似古时的四大美人一样闭月羞花,遥望白茫茫的一片,好似花的海洋,风轻轻吹过,花瓣纷纷落下,给小路盖上了花地毯。

\section{2022年3月22日 星期二 雨}

杨若君

今天我走在放学回家的路上,这时正下着小雨。天阴沉沉的,我觉得下雨仿佛就带着一种心情——伤感,压抑。但我仔细地思考了一番,我又觉得下雨也不是特别不好。细雨从天空中轻轻地落下来,落到人们的伞上和地上,发出“滴嗒”的声音,从伞上滑下来,又掉到地上,好像在跳舞,又好像在唱歌,风儿“沙沙”作响,好像在为雨点儿伴奏。

下雨好像也别有一番乐趣!

\section{2022年3月22日 星期二 雨}

陈姿伊

今天的雨依旧下得很大,回家的路上,我虽撑着伞,但是裤角和袖还是湿了。雨伞从树叶下蹭去,发出沙沙的声响,雨水顿时如同帘子般从我眼前洒下,滴湿了我的鞋。雨哗啦啦地下个不停,路上的行人都低着头,撑着伞,小心翼翼地躲避着地上的水坑;但顽皮的小孩子穿着各式各样的小雨衣,在水抗里踩着,跳着……

\section{2022年3月23日 星期三 晴}

李朗

唉,又轮到我写闲言碎语啦。虽然说是“闲言碎语”,但是“闲言”和“碎语”我也没有多少啊!写什么呢?我一边用笔的尾巴高敲着脑袋,一边望向窗外。窗外阳光明媚,但这阳光又有一点刺眼。我想不出写啥,只好看了看之前张欣瑜和杨若君写的,她们两个,一个写樱花,一个写雨,写得都很好,都得了五星。我又不是她们,当然写不到这么好。而且,我写花写草那可十分令人头疼。写什么才能得五星呢?真令人头疼,难道我就得不了五星吗?

\section{2022年3月23日 周三 晴}

姚琪

“天下强人无处不有,没谁能躲跑步的手”,这不一听到跑800米,同学们都想了一堆点子:有的借“上厕所”为理由不想跑;有的说“昨天脚扭伤了”不能跑;有的说“临时肚子痛,跑不动”…… 哎!这理由一个比一个高明呀。可该来的终究都得来呀,于是我也随大流去跑了。1分16……2分10……3分20,呼…… 还好是优秀。好了好了,不说了,我要静静休息一下…… 咳咳!

\section{2022年3月23日 星期三 多云}

戴瑞彤

“叮铃铃!”上课铃响了,华老师的脚步也随之而来,这节课是体育课,并且要跑800米。大家一听见这个消息,哀嚎一片。有的人以上厕所为由,想在厕所里消磨时间,不想妥协;有的人则以脚扭了、肚子痛为由,想在一旁休息,但都被华老师拒绝了。没办法,只好硬着头皮上了,我缓步走上跑道,咽了口唾沫。“预备,跑!”随着一声哨响,同学们都跑了起来。我一开始冲得很快,跑在了第一个,但800米和100米可是不一样的,它的路程可比100米长多了,不过一会儿,我就被三个同学超过了。一股风吹来,吹得小草左右摇摆。不行,我不能就这样任由别人超过,我在心里暗暗想着,可我腿不知怎么的,有些抬不起来。一圈…… 两圈…… 到了最后一圈,我的腿好酸,喉咙里也有种喘不过气的感觉。100米……50米……,我离终点越来越近,我开始用尽最后的力气向前冲去。

“啪”我的脚重重地踩在了终点线上,3分20秒,我达标了!

\section{2022年3月26日目星 星期六 多云}

李卓涵

昨天,马晨轩说要请我吃雪糕。选好了,他便带我去付钱。网很卡,他在点数字“2”时没点出来。这下马晨轩有点儿生气了。他一边嘟囔着:“这手表咋回事?”一边狂点数字“2”。忽然一个“2”跳了出来。马晨轩刚要点“支付”,数字忽然快速跳动,成了“22222”,马晨轩差点儿就点着了(他的手表支付宝连着他妈妈的),可把他吓了一大跳。他差点儿付了一笔巨款。

\section{2022年3月28日 星期一 阴}

张欣瑜

清晨时,天空是最蓝的,晴朗的天空没有云,给了蓝色最大的发挥空间,毫无保留地把蓝色渗透出来,晕满整个天空。这样的蓝天与花交相辉映,是最灿烂的色彩,最纯净的色彩。澄碧的蓝天上飘着缓缓流云,凉爽的清风中,清晨更加深沉优美。

\section{2022年3月29日 星期二 晴}

杨若君

今天的天气特别好,太阳正笑咪咪地看着我。中午,妈妈给我做了饼,我看了看桌上的饼说:“哎,怎么天天都是饼呀,能不能换一个吃呀?”妈妈拿我没办法,只好再去做。而这时,我坐在桌前,看着桌上的饼,再看看在厨房里的妈妈,我想自己是不是做地太过分了。过了一会儿,妈妈又给我端来了一碗裹满鸡蛋的炒饭,我心中不禁感到无比愧疚,我的泪水涌入眼眶。而这或许就是母爱吧。

\section{2022年3月29日 周二 晴}

陈思涵

上周末在江边散步,正走着,一个小孩拿着一根棒棒糖,从我和朋友身边飞奔而过。他刚走没一会儿,我又遇到了一位约莫20来岁,刚毕业的姐姐骑着一辆自行车,车篮里还放着一沓书。我和朋友笑着继续往前走,却遇到了一位白发苍苍的老奶奶,她看上去已经很老了,灰白的发丝挂在脸颊上,盖住了一条条岁月留下的刻痕。她坐着轮椅,慢悠悠地往前推动着轮子。望着她那离去的背影,我不禁有些恍了神:“岁月原来真的只是匆匆过客啊!前面遇到的那些人好像就代表每个年龄诶——从一个笑着奔跑的孩子,变成一个上学的学生,一个刚入职的新人,再到一个中年人,慢慢就成为了一个坐着轮椅的老人。唉,真是岁月不待人啊!

\section{2022年3月30日 星期三 晴}

李朗

早上我起来的时候,感觉之前一直在摇的牙齿变得更摇了,十分地高兴,便跳下床去,冲到洗手间,用镜子照着,看看大概什么时候能掉。我晃了晃它,可它十分坚毅,不肯轻易松开。看来还要很久才能掉,我的心又跌落到了深渊。中午,午餐有一道梅干菜和肉,在吃梅干菜的时候,只听见咯吱咯吱两声,我吓了一跳,赶忙用舌头穿过梅干菜堆去碰那颗顽固的牙。我的舌尖很快碰到一个有点摇摇欲坠的牙,我放下心来,咽下了梅干菜。午饭后,无聊的我又去摇那颗牙,突然感觉它已经要掉了!我又吓了一跳,默默估算它能不能撑过今天,估计够呛我想着。无奈使我不得不决定现在把它拔掉。我来到卫生间,干净利落地把它拔了。呼,我松了气,这颗牙真让我煞费苦心啊!

\section{2022年3月30日 星期三 阴}

丁凝

偶然在一个背阴的树坑里看见了一抹惊鸿,仔细看竟然是朵蒙了尘的郁金香,这原本应该是被人呵护的花,却委屈地被落在了泥坑里,尽还顽强地存活了下来。你看那金黄色的花瓣,不因蒙尘而失去花的光泽,反而悠然自得,随遇而安,与细长的叶片在风中自在的摇曳,这朵花真的是受了委屈吃亏了吗?

\section{2022年3月31日 星期四 晴}

盛逸宸

“阳光体育”又被称作“阳光地狱”,是我最不喜欢的一节课,因为它刚开始时的跑步实在是太累了。就以今天为例,最开始的一两圈速度还算正常,我也能跟上大部队,但慢慢的四圈、五圈、六圈…… 不知道是大部队太快了还是我跑的太慢了,我逐渐脱离了大部队,但跑圈仍在继续。我卖力地跑着,但始终追不上前面的同学。终于,在我已经被大部队丢了大半圈的情况下,跑圈结束了。我坐在地上,喘着粗气,久久不能平复。

\section{2022年3月31日 星期四 雨转阴}

丁凝

路过操场,有两只斑鸠在其上玩耍。操场上空无一人,只有树叶在簌簌地抖动。斑鸠愉悦的鸣叫着,一只散着步,一只不停的盘旋,时不时出现互换。我猛地冲过去,他们似乎吃了一惊,都停着不动,偏着脑袋看着我,似乎忘记了飞是怎么一回事儿,直到我差点踩在他们身上,他们才眨眨眼,慌慌张张的,没几步,扑扇翅膀飞走了。
