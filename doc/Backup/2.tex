\chapter{第二期}

\section{2019年6月3日 星期一 晴}

余蕙琳

今天放学和杨若君回家时,遇见了我们班的戏精:李恩博。为什么说他是戏精呢?看了这篇文章你就知道。这会儿他从书包里拿出水杯,像孙悟空一样抡起水杯就向杨若君打。``呀!救命呀!''杨若君慌忙而逃。可现在他又把书包挂到手腕上,装出书包很重的样子。嘴里还喊着``重死了,重死了!''可他还蹦嘣跳跳地走着。走到半路,他不再往前走,而是一个180度急转弯向另一个方向去了,还像豹子一样冲向马路,他不说书包重吗?怎么又变成这个样了。哎,真是个戏精。

\section{2019年6月3日 星期一 雨}

为了看杭州落日的绮丽景色,我吃完饭就来到江边。这时候,太阳已经西斜了,它像个害羞的姑娘露出那红彤彤的脸蛋,周围绕着一圈光环。那天空是淡蓝色的,像明净的湖水,现在越来越蓝,越来越浓了,简直有点儿红了呢!落日还像喝了酒的红脸醉汉,跌落在山那边,江边连绵起伏的群山在夕阳的照射下像披上了闪闪发光的金衣。这时候,落日把水和天映得一半通红,一半金黄,看起来分外壮观。我情不自禁地赞叹:``啊!杭州的落日可真迷人!''

\section{2019年6月3日 星期一 晴}

叶雨飞

今天下午,殷教练让我和史佳卉双打严秋实和李卓涵。这也太不公平了!我们俩都在想,第一,他们是男孩,而我们是女孩;第二,他们全是09年的,而我们全是10年的。殷教练又说:``男生如果输了,跑两组步伐!''我们松了一口气。比赛开始了!我们心里都想着,千万要赢啊!前半局,我和史佳卉配合得相当好,领先了6分。后来,越战越勇,你一分我一分,最后大胜。嘿嘿,我们兴高采烈地看着他们跑步,他俩则愁眉苦脸的。

\section{2019年6月4日 星期二 晴}

付梦欣

放学回家的路上,我一边走一边回想学校里的趣事。突然,一个不明之物``嘭''的一声掉在了我的脑门上,又清脆地掉到了地上。``这是什么鬼东西啊!''我尖叫道。抬头望去,交叉的树枝和稠密的树叶遮住了太阳,李子一颗挨着一颗躲在树叶下乘凉。近看,李子就像一颗颗闪闪发光的红宝石,点缀着李树。远看,沉甸甸的``红宝石''把树枝压弯了腰,好像在向路人们鞠躬问好,炫耀它的硕果。

\section{2019年6月5日 星期三 晴}

许智涵

今天,当詹老师宣布今天的体育课换成语文课时,同学们都唉声叹气。但又马上追问:``詹老师,那体育课换到什么时候了?''老师回答:``明天的第一节语文课换成体育!''同学们听到这消息,兴奋得大喊:``哇!耶!\ldots\ldots{}''我转头一看,陈浩然竟然高兴得发了疯,跳起了``吃鸡舞''!看来,体育课真的是最受欢迎的一门课了!

\section{2019年6月10日 星期一 暴雨}

叶雨飞

今天下午,我们正在订正数学作业,忽然,我听见一阵哗啦啦的声音,扭过头一看,只见外面下着瓢泼大雨(或者说倾盆大雨),未很惊奇:刚才不是还晴天吗?看来老天爷变脸也很快啊!我心想:糟糕,待会儿去练羽毛球,不会被淋成落汤鸡吧?但马上又想起来今天值日,希望值日值晚点,说不定雨就停了!后来,等我们值日完,天又晴了。还好,没被淋成落汤鸡!

\section{2019年6月13日 星期四 雨}

陈奕铖

今天早上,叮铃铃一阵闹钟声把我在睡梦中惊醒。原来是我的老朋友闹闹在叫我起床,我伸了伸懒腰。说:``闹闹你别吵了,现在是6点,我的鸡腿还没吃完呢!''闹闹一直叫``叮铃铃、叮铃铃\ldots\ldots{}''我马上起来把闹闹拿到床上,迷迷乎乎的一边关闹闹,一边说:``闹闹还我鸡腿\ldots\ldots,''

\section{后记}

同学们从今天开始就开始放暑假了,大家是不是很激动呢?

过去的一个月,在詹老师,各位同学和各位家长的努力下,《闲言碎语》活动继续坚持着。詹老师每天要花大量的时间批阅每位同学们的笔记,感谢老师的默默付出。同学们更是辛苦,在完成作业和各种培训班的空余时间还要绞尽脑汁一篇日记,确实不容易。家长们也积极响应,百忙之中抽时间把孩子的文章打印成电子稿发过来。总之,在大家的共同努力下,有了我们的闲言碎语第二期。

闲言碎语的初衷是鼓励大家能坚持写作,提高写作水平。写作是一项非常有价值的技能,遗憾的是我上学时经常语文不及格,写作水平不好,常为此而遗憾。前几天听吕丽娟老师的讲座说:``执笔人有历史解释权''。意思是说,后人眼中的事实,是由前人笔下的文章所描绘的。所以,能写好文章是一件神圣的事情。如果读到一篇与自己想法一致的文章,作者写出了你想要说的话,那是多么地畅快。希望大家能体会到写作的重要性,不仅仅是为了应付考试,更是为了能更好的表达自己。

在公众号发布的每日之最,只是为了鼓励同学们的积极性,同时希望通过这样一个仪式,鞭策大家能一起坚持下去。没有选中的同学切莫灰心,家长也切勿打消他们的积极性。此时正值芬兰的春夏之交,路边长满了各种野花草。前几日遍地金黄的蒲公英铺洒在绿色的草地上,在绿色的映衬下黄得那么耀眼。没过几日蒲公英全白了,不知不觉间又冒出斑斑点点的,说不上名字的白花、黄花点缀在更茂盛的草丛中。也许在人们看来金黄的蒲公英是最漂亮的。但是在花花草草看来,他们都在享受着阳光的温暖,竭力地展现自己最美好的时光,结下丰硕的种子。

另外,想跟大家分享写作方法。我不是专家,所以不敢胡言乱语,且引用林语堂先生关于写作的建议。第一,写作如作画,可分为临帖和写生。临帖就是要多看书,模仿前辈的方法进行写作。本期中一位同学描写自然景物的文字,观察细致入微,文字生动优美。定是她平时积累的写作素材比较多。写生就是自己创作了。第二,初级阶段可以以描写文和叙事文为主。他认为因为阅历比较少,刚开始比较难以写出议论文。描写文和叙事文写好了以后,写议论文,也就功到自然成了。本期中大多数同学写的就是叙事文,其中有位同学讲述了家中的一件小事,他在写的时候把自己感动得眼泪吧嗒吧嗒直落,旁人读了,也深为感动。请相信,能感动自己的事,也一定能感动他人。所以,写出心中所想,记下感动自己的事,也一定能感动读者。

最后,祝大家暑期愉快,四年级再见。
