\chapter{后记}

闲言碎语写作活动从2019年开始,不知不觉已经进行了三年,大家马上要小学毕业了,接下来可能要到不同的中学或不同的班级就读,能再次成为同班同学就只能靠缘分了。不论到何时,不论到哪里,如果当你拿起这本小册子,都能回想起在闻涛小学和同学,和老师,和家长一起渡过的六年时光,将是这本小册子最大的欣慰。

在“闲言碎语”活动开启最初,我们没想到它能印刷成书。首先得感谢你们自己,是每位同学笔耕不辍的坚持,才能积累到这么多鲜活生动的文字。再得感谢詹老师,在她的鞭策鼓励下,每位同学才有不断坚持的决心;詹老师每天得花大量时间来批阅大家写的稿子,这本小册子里收录的内容只是冰山一角,老师在背后的付出远远比我们看到的多得多。还得感谢叶慕妈妈为本书画的充满童趣的插图。最后得感谢各位同学家长的支持,大家在百忙之中抽时间将手写稿转成电子稿发给我们。感谢每一位的用心付出,是大家共同的努力才让这本书成为了可能。

这本小册子的扉页上写着:“献给起点的回忆”。它是送给每位同学人生起点的回忆,文化起点的回忆,也是送给每位家长初为父母的回忆。十年,二十年之后你还能找到这本小册子吗?你会不会再读一遍?会不会想起那段时光呢?会不会想起曾经伴我们一起走过那段路的人呢?能够用文字记录自己的经历、感受,当我们再读这些文字,又能复现往日的经历,被文字所感动,这不就是写作重要作用之一吗?

写作不仅仅是为了考试能写出好的作文。写作是一项重要的技能,我们可以通过写作表达自我,观照内心。停下的脚步看看路边的小草,也许我们会赞叹生命的奇迹;抬起头仰望天上的浮云,也许我们会想起远方的家乡。见花见叶均有意,观山观水皆有情,只要用心去感受周遭的一切都会让人心生感悟,就有了呼之欲出,不吐不快的想法,我们可以拿起笔记录下这些心情。大家回想一下,在学习过程中,书写文字的数量可以一个星期用掉一只笔芯,可是我们有拿起笔给亲人、朋友写过一封信,分享我们的经历吗?我们有拿起笔记录下自己成长的心情吗?

我们每天被大量的文字,视频所淹没,可是这些内容跟我们自己有多大的关系?这些信息有多少是真实的?发布这些新闻,视频的人认同他们自己说的话吗?这些信息就像从工厂里生产出来的塑料花,它们色彩鲜艳却毫无生机,每一朵都是一模一样的工艺品,甚至是有毒的垃圾。这段时间爆出某位知名儿童文学作家的书内充斥着大量的垃圾内容,引起家长共愤。我们需要反思下自己,作为家长,我们看了给孩子买的书吗?难道仅仅因为它是权威专家推荐的就不加分辨地一定是好的内容?这些作家他们是认真地在写书吗?一位作家在一年之内出的书摞起来近1米高,这种垃圾食品还是不要吃,更不要给孩子吃。那些专家读过他们推荐的书吗,如果读过,书里出现大量粗俗文字,丑陋不堪的插图他们看不出来吗。所以,我们要凭自己的内心独立去判断。
那些用心下的文字才是有勃勃生机的自然之花,它有无限的细节,值得长久地品评,它是能引起共情,摇动人心。时间是一把筛子,它终究会去芜存真,符合内心感受的才能永久流传。

从我做起,从现在做起,拿起笔,记录下自己的过往,书写下自己的当下,描绘自己的梦想。哪怕文笔不好,哪怕没有读者,哪怕在阅卷老师那里拿不到高分,只要我们用心去观察,用心去感受,我们会在文字中找到自我,找到自己的未来。每一位中国人的从心出发,才能迎来民族的伟大复兴,每一位中国人的未来,就是民族之未来。

\vspace{5em}

杨建林

2022年6月3日